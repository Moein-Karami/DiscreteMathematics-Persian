\SECTION{مقدمه ای بر استقرای ریاضی}

 در علوم تجربی و عالم طبیعت، وقتی قانونی با مشاهده کردن و آزمایش کشف می‌شود، باید با آزمایش های متعدد بعدی، در شرایط مشابه، اعتبار سنجی شود؛
 اما همه از دوران دبستان به خاطر داریم که در ریاضیات داستان کمی متفاوت است! 

 اکثر وقت ها با بازی کردن با اعداد و ارقام و نظم دادن به آنها، توانسته ایم که به روابط مختلف حاکم بر عالم ریاضیات، پی ببریم و آنها را اثبات کنیم.
 در این بین، با روش های مختلف اثبات نیز آشنا شده ایم که حساب تعداد آنها از دست در رفته است!
 حال سوالی که باید پرسید، آن است که آیا علوم تجربی و ریاضی همیشه انقدر از هم‌دیگر فاصله دارند؟
 
 استقرای ریاضی، روشی  برای اثبات همان مسائل همیشگی ریاضی است با چاشنی علوم تجربی!
 در این روش، سعی ما این است که با مشاهده قوانین حاکم بر یک مجموعه، آن قوانین را برای مجموعه های مشابه نیز اثبات کنیم.
 به عبارت دیگر، تعداد اعضای مجموعه هایی که چنین قوانینی در آنها صدق می‌کنند را آنقدر افزایش دهیم، تا این قوانین در تمام مجموعه های فرضی نیز، صادق باشند.
 دقت کنید که همانطور که گفتیم، استقرای ریاضی مرزی بین عالم خیال و تجربه است؛ به همین علت
  واضح است که در هر کجای این فصل هنگامی که صحبت از اثبات مجموعه‌ای از حالات می‌شود، منظور مجموعه‌ای از اعداد طبیعی (قابل مشاهده) می‌باشد.
  به عبارتی ما نمی‌توانیم (و نیازی نداریم) که یک قانون را برای یک مجموعه $2.5$ عضوی و یا 
  $\sqrt{3}$
  عضوی ثابت کنیم!

  اگر علاقه دارید که در رابطه با پیشینه استقرای ریاضی و اولین کاربردهای آن، بیشتر مطالعه کنید، 
  به ماهنامه ریاضی آمریکا سر بزنید.
  \footnote[1]{The Origin of Mathematical Induction. By W. H. Bussey}