\begin{PROBLEM}
    \p
    ثابت کنید اصل استقرای ریاضی برقرار است، اگر و تنها اگر اصل خوش ترتیبی برقرار باشد.

    \SOLUTION{
        \p
        مساله را به دو بخش تقسیم می‌کنیم.
        
        \begin{enumerate}
            \item اگر اصل خوش ترتیبی برقرار باشد، اصل استقرای ریاضی نیز برقرار خواهد بود:

                مجموعه
                $\mathnormal{A}$
                را در نظر بگیرید؛ فرض کنید دو شرط اصلی استقرا برای آن برقرار است؛ یعنی
                $1 \in \mathbb{A}$
                و برای هر
                $\mathnormal{n}$
                طبیعی، اگر
                $n \in \mathnormal{A}$
                آنگاه
                $n + 1 \in \mathnormal{A}$،
                حال باید با استفاده از اصل خوش ترتیبی ثابت کنیم:
                $\mathnormal{A} = \mathbb{N}$

                مجموعه 
                $B$
                را مجموعه اعداد طبیعی در نظر بگیرید که متعلق به
                $A$
                نیستند.
                اگر
                $B$
                تهی باشد که مساله حل است. در غیر این صورت، با برهان خلف اثبات را ادامه می‌دهیم.
                
                فرض خلف: مجموعه
                $B$
                ناتهی است.
                طبق اصل خوش ترتیبی، این مجموعه دارای عضوی مینیمم مانند
                $\mathnormal{k}$
                است. از آنجایی که 
                $1 \in A$
                پس:

                \begin{flushleft}
                    $\mathnormal{k} \neq 1 \Rightarrow \mathnormal{k - 1} \geq 1$

                    $\mathnormal{k - 1} \notin B \Rightarrow \mathnormal{k - 1} \in A$
                \end{flushleft}
                
                اما حال طبق شرط دوم مجموعه
                $A$
                می‌دانیم که اگر
                $k - 1 \in A$
                آنگاه
                $k \in A$
                که با فرض 
                $k \in B$
                در تناقض است! پس فرض خلف باطل است و مجموعه
                $B$
                تهی است و داریم: 
                $A = \mathbb{N}$
            
            \item اگر اصل استقرای ریاضی برقرار باشد، اصل خوش ترتیبی نیز برقرار خواهد بود.
                مساله را با برهان خلف حل می‌کنیم؛ مجموعه
                $A$
                را مجموعه‌ای ناتهی از اعداد طبیعی در نظر بگیرید که عضو مینیمم ندارد.
                حال مجموعه 
                $B$
                را به شکل زیر تعریف می‌کنیم:

                \begin{flushleft}
                    $B = \{x \in \mathbb{N} | x \leq a; \forall a \in A\}$
                \end{flushleft}

                از آنجایی که
                $A$
                عضو مینیمم ندارد، پس
                $A \cap B = \emptyset$؛
                از طرفی دیگر، واضح است که
                $1 \in B$. 
                حال فرض کنید
                $k \in B$
                باشد، پس هر عدد طبیعی کوچک‌تر مساوی
                $k$
                نیز باید کوچک‌تر مساوی تمامی اعضای
                $A$
                باشد؛ پس طبق تعریف داریم:

                \begin{flushleft}
                    $1, 2, 3, \ldots, k \in B$
                \end{flushleft}

                و چون
                $A \cap B = \emptyset$    
                پس: 
                         
                \begin{flushleft}
                    $1, 2, 3, \ldots, k \notin A$
                \end{flushleft}

                حال اگر
                $k + 1 \in A$
                آنگاه
                $k + 1$
                کوچک‌ترین عضو
                $A$
                خواهد بود؛ پس
                $k + 1 \notin A$
                و در نتیجه:

                \begin{flushleft}
                    $1, 2, 3, \ldots, k, k + 1 \notin A \Rightarrow 1, 2, 3, \ldots, k, k + 1 \in B$
                \end{flushleft}

                پس از 
                $k \in B$
                نتیجه گرفتیم
                $k + 1 \in B$.
                حال چون هم شرط پایه و هم شرط گام استقرا برقرار هستن، می‌توانیم نتیجه بگیریم:
                $B = \mathbb{N}$.
                بنابراین مجموعه
                $A$
                تهی خواهد بود که با فرض خلف در تناقض است. پس حکم ما ثابت می‌شود.


            \end{enumerate}
    }

\end{PROBLEM}

\begin{PROBLEM}
    \p
    با استفاده از اصل خوش ترتیبی نشان دهید
    $\sqrt{2}$
    گنگ است.

    \SOLUTION{
        \p
        مساله را با برهان خلف حل می‌کنیم.
        فرض کنید 
        $\sqrt{2}$
        گنگ نباشد. پس طبق تعریف، دو عدد حقیقی
        $p, q \in \mathbb{Z}, q\neq0$
        وجود دارند به طوری که
        $\frac{p}{q} = \sqrt{2}$
        حال با کمی جا‌به‌جایی کسرها داریم:
        
        
        \begin{flushleft}
        $\frac{p}{q} = \sqrt{2} \Rightarrow p^2 = 2q^2 \Rightarrow p^2 - pq = 2q^2 - pq$

        $\Rightarrow p(p - q) = q(2q - p) \Rightarrow \frac{p}{q} = \frac{2q - p}{p - q}$
        \end{flushleft}
        
        دقت کنید که از انجایی که 
        $q \neq 0$
        و
        $q \neq p$
        پس مخرج‌ها صفر نمی‌شوند و به مشکلی بر نمی‌خوریم.

        از طرفی دیگر، می‌دانیم 
        $p$
        و
        $q$
        هم‌علامتند؛ پس تعریف می‌کنیم:

        \begin{flushleft}
            $m = |p|$

            $n = |q|$
        \end{flushleft}
            
        حال 
        $\mathnormal{A}$
        را مجموعه تمام
        $n \in \mathbb{N}$
        در نظر می‌گیریم. طبق اصل خوش‌ ترتیبی، این مجموعه یک عضو مینیمم دارد که آن‌ را
        $n^\prime$
        می‌نامیم.
        عضو متناظر
        $n^\prime$
        را نیز
        $m^\prime$
        می‌نامیم. پس داریم:

        \begin{flushleft}
            $\frac{m^\prime}{n^\prime} = \sqrt{2} = \frac{2n^\prime - m^\prime}{m^\prime - n^\prime}$
        \end{flushleft}

        پس طبق تعریف 
        $m^\prime - n^\prime$
        نیز یکی از اعضای مجموعه
        $\mathnormal{A}$
        می‌باشد.
        اما می‌دانیم:
        
        \begin{flushleft}
            $m^\prime = \sqrt{2}n^\prime \Rightarrow m^\prime < 2n^\prime \Rightarrow m^\prime - n^\prime < n^\prime$
        \end{flushleft}
         
        پس عضوی کوچک‌تر از 
        $n^\prime$
        در مجموعه
        $\mathnormal{A}$
        پیدا شد! پس به تناقض برخوردیم و فرض خلف باطل و حکم برقرار است.
    }

\end{PROBLEM}