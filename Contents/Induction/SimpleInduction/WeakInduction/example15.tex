\begin{PROBLEM}
	\p 
	%منبع :‌روش های ترکیبیات ۲
	یک مربع با ابعاد واحد داریم. مرکز یک دایره به شعاع 
	$r > 0$
	داخل این این مربع وجود دارد(مرکز این دایره می‌تواند روی اضلاع نیز باشد).
	همچنین یک نقطه نیز داخل این مربع موجود (داخل مربع است نه روی اضلاع) است.

	در هر مرحله می‌توان یکی از رئوس مربع را انتخاب کرد و سپس این نقطه را به وسط خط واصل این نقطه و آن راس انتقال داد.
	ثابت کنید می توان تعدادی از این عملیات این نقطه را به داخل دایره انتقال داد.

	\SOLUTION{
		\p
		در ابتدا این طور به نظر می‌رسد چون 
		$r$
		یک عدد پیوسته است نمی‌توان در این سوال از اسقرا استفاده کرد. برای حل این مشکل ابتدا حکم سوال را تغییر می‌دهیم:

		کوچک ترین عدد صحیح مانند 
		$n$
		را در نظر بگیرید که 
		$2^{-n} \leq r$.
		سپس یک دایره به شعاع 
		$2^{-n}$
		به مرکزیت دایره قبلی بکشید.
		حال حکم را به این صورت تغییر می‌دهیم که اثبات کنید می‌توان نقطه را به داخل دایره جدید منتقل کرد. واضح است
		که چون دایره جدید داخل دایره اصلی است پس با اثبات این حکم، حکم اصلی نیز اثبات می‌شود.

		حال حکم را با استقرا روی 
		$n$
		اثبات می‌کنیم.

		واضح است که حکم به ازای 
		$n < 0$
		صحیح است چون کل مربع داخل دایره است در نتیجه نقطه از همان ابتدا در دایره قرار دارد.

		حال فرض کنید حکم به ازای 
		$n = k, -1 \leq k$
		صحیح باشد، ما حکم را به ازای 
		$n = k + 1$
		اثبات میکنیم.

		واضح است که فاصله ی مرکز دایره از یکی از رئوس مربع حداکثر نیم واحد است.
		حال دایره را نسبت به این راس و با ضریب 
		$-2$
		قرینه می‌کنیم. نتیجه این کار یک دایره به شعاع 
		$2^{-k}$
		است که مرکز آن داخل مربع است. حال طبق فرض استقرا می‌توان نقطه را به این دایره انتقال داد، بعد از این کار فقط کافی است
		که فاصله‌ی نقطه را با راس مذکور مربع نصف کنیم، با این کار نقطه وارد دایره اولیه می‌شود. در نتیجه حکم استقرا به ازای
		$n = k + 1$
		اثبات می‌شود.

		پس حکم استقرا به ازای هر
		$n \in Z$
		صحیح است و حکم سوال اثبات می‌شود.
	}
\end{PROBLEM}