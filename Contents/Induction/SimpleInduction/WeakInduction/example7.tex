\begin{PROBLEM}
	\p
	دنباله
	$a_n$
	به این صورت ساخته می‌شود:
	$a_{n + 1} = 3a_n^4 + 4a_n^3, n \in N$
	نشان دهید که نمایش دهدهی 
	$a_10$
	حداقل
	$1000$
	رقم 
	$9$
	دارد.

	\SOLUTION{
		\p
		برای به دست آوردن شهود دو عدد اول را می‌نویسیم
		$a_0 = 9, a_1 = 22599$
		با توجه به این دو عدد حدس می‌زنیم که نمایش دهدهی 
		$a_i, i \in W$
		در انتهای خود حداقل 
		$2^n$
		رقم 9 دارد.
		
		حال این حدس خود را به کمک استقرا روی
		$n$
		اثبات می‌کنیم.

		حکم استقرا: 
		$a_i, i\in W$
		در انتهای نمایش دهدهی خود حداقل 
		$2^n$
		رقم 9 دارد.

		حکم به ازای 
		$n = 0$
		صحیح است.

		حال فرض کنید حکم به ازای 
		$n = k; k \in W$
		صحیح باشد. ما حکم را به ازای 
		$n = k+1$
		اثبات می‌کنیم.

		در ابتدا توجه کنید اگر عددی مانند 
		$x\in N$
		در انتهای نمایش دهدهی خود شامل 
		$z$
		تا رقم 9 باشد داریم:
		$x = y * 10^z - 1; y \in N$
		در نتیجه طبق فرض استقرا داریم: 
		$a_k = y * 10^z - 1; z = 2 ^ k,y \in N$
		در نتیجه
		$a_{k+1} = 4a_k^3 + 3a_k^4 = 4(y * 10 ^z - 1)^3 + 3(y * 10^z - 1)^4 = 3y^4 10^{4z} - 8 y^3 10^{3z} + 6 y^2 10 ^(2z) - 1$
		$ = t 10^{2z} = t 10^{2^{k+1}}; t \in N$

		پس حکم استقرا به ازای 
		$n = k+1$
		نیز صحیح است پس حکم به ازای هر 
		$n \in W$
		صحیح است.

		حال با قرار دادن 
		$n = 10$
		متوجه می‌شویم 
		$a_10$
		حداقل 1024 رقم 9 در انتهای خود دارد پس حکم سوال اثبات می‌شود.
	}
	%استراتژی های حل مسئله ص۲۷۷
\end{PROBLEM}