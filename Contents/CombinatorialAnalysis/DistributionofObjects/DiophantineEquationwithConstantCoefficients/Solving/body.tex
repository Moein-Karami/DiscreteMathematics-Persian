\begin{PROBLEM}[تعداد جواب‌های معادله سیاله خطی با ضرایب واحد در مجموعه اعداد حسابی]
    \p
    $$\sum\limits_{i=1}^n x_i = X$$
    معادله سیاله بالا چند جواب دارد
    اگر:
    $$x_i \in \mathbb{W}, 1 \leq i \leq n$$

    \SOLUTION[پاسخ اول]{
        \p
        مسئله را به تقسیم
        $X$
        مهره نامتمایز
        به
        $n$
        جعبه متمایز مدل می‌کنیم.
        ادعا می‌کنیم مسئله معادل چینش با ترتیب
        $X$
        مهره و
        $n-1$
        مداد است.
        می‌دانیم
        $n-1$
        مداد به صف شده،
        $n$
        فضای متمایز تشکیل می‌دهند
        (بین هر دو مداد و قبل از اولین مداد و بعد از آخرین مداد).
        بنابراین می‌توانیم هر یک از این فضاها را به یک جعبه نسبت دهیم
        و مهره‌های قرار گرفته در هر فضا را درون جعبه‌ی متناظر قرار دهیم.
        پس این دو مسئله یکسان هستند.
        طبق قضیه
        \CROSSREF{تعداد جایگشت‌های خطی متمایز با اعضای تکراری}،
        پاسخ این مسئله برابر است با:
        $$\frac{(X+(n-1))!}{X!(n-1)!}$$
    }
        \SOLUTION[پاسخ دوم]{
        \p
       لازم داریم
    $X$
    مهره و
    $n-1$
    مداد ذکر شده در مدلسازی پاسخ اول را در
    $X+n-1$
    جایگاه متمایز، جایگذاری کنیم
    (بیانی دیگر برای به صف کردن).
    ابتدا، از بین این
    $X+n-1$
    جایگاه،
    $X$
    جایگاه را برای مهره‌ها انتخاب کرده و مهره‌ها را درون آن‌ها قرار می‌دهیم.
    سپس مداد‌ها را در جایگاه‌های باقی‌مانده قرار می‌دهیم و ادامه مدل‌سازی را مانند آنچه در پاسخ اول آمده، دنبال می‌کنیم.
    بنابراین، تعداد حالات انجام این کار برابر
    ${X+n-1 \choose X}$
    خواهد بود.
    }
\end{PROBLEM}

\p
دو پاسخ بالا ارتباط میان 
\CROSSREF[جایگشت خطی با اعضای تکراری]{جایگشت خطی با اعضای تکراری}
و 
\CROSSREF{ترکیب چندگانه}،
که در بخش 
\CROSSREF[جایگشت خطی با اعضای تکراری]{ارتباط میان جایگشت خطی با اعضای تکراری و ترکیب چندگانه}،
به صورت مفصل به آن پرداخته شده بود را نشان می‌دهد.

\begin{THEOREM}
    \TARGET{تعداد پاسخ‌های معادله سیاله خطی با ضرایب واحد در مجموعه اعداد حسابی}
    \p
    به صورت کلی، تعداد پاسخ‌های معادله سیاله خطی با ضرایب واحد در مجموعه اعداد حسابی،
    با $n$ متغیر و مقدار ثابت $X$ برابر است با:
    $$\frac{(X+(n-1))!}{X!(n-1)!} = {X+n-1 \choose X}$$
\end{THEOREM}


\begin{PROBLEM}[محدودیت حداقلی متغیر‌ها در معادله سیاله خطی با ضرایب واحد]
    \TARGET{محدودیت حداقلی متغیر‌ها در معادله سیاله خطی با ضرایب واحد}
    \p
    $$\sum\limits_{i=1}^n x_i = X$$
    معادله سیاله بالا چند جواب دارد
    اگر:
    $$x_i \in \mathbb{W}, x_i \geq t_i; 1 \leq i \leq n$$

    \SOLUTION{
        \p
        تغییر متغیر زیر را درنظر بگیرید:
        $$y_i = x_i - t_i; 1 \leq i \leq n$$
        بنابراین می‌توانیم معادله صورت سوال را به شکل زیر بازنویسی کنیم‌:
        $$\sum\limits_{i=1}^n x_i = \sum\limits_{i=1}^n (y_i + t_i) = X$$
        $$=> \sum\limits_{i=1}^n y_i = X - \sum\limits_{i=1}^n t_i$$
        \p
        برای درک بهتر مفهوم تغییر متغیر بالا، مسئله را با یک شبیه‌سازی تعبیر می‌کنیم.
        شرایط ذکر شده در این مسئله مانند آن است که در مسئله مهره و جعبه که در سوال قبل ذکر شد،
        برای هر جعبه، تعداد حداقلی تعیین شود که حتما به آن تعداد مهره در آن جعبه قرار گیرد.
        راهکار حل این مسئله آن است که در ابتدا به تعداد خواسته شده، مهره درون آن جعبه‌ها قرار گیرد
        و سپس با مهره‌های باقیمانده، مسئله به یک مسئله معادله سیاله مدل شود.
        به این صورت تعداد
        $\sum\limits_{i=1}^n t_i$
        مهره از 
        $X$
        مهره‌ای که داشتیم کم شده و با تعداد باقیمانده مسئله را ادامه می‌دهیم.
        \p
        حال مطابق با پاسخ سوال قبل، پاسخ مسئله برابر است با:
        $$\frac{(X - \sum\limits_{i=1}^n t_i)+(n-1))!}{(X - \sum\limits_{i=1}^n t_i)!(n-1)!}$$
    }
\end{PROBLEM}