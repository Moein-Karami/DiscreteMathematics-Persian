        \p
عبارت
$$(x^3 + x^4 + x^5 + \cdots)^3$$
رابه شکل زیر تغییر می‌دهیم:
$$(x^3(1 + x + x^2 + \cdots))^3 =$$
$$x^9(1 + x + x^2 + \cdots)^3 =$$
$$x^9\frac{1}{(1-x)^3}$$
ضریب
$x^{20}$
در عبارت داده شده برابر است با ضریب
$x^{11}$
در بسط
$\frac{1}{(1-x)^3}$
. از طرفی می‌دانیم که شریب
$x^r$
در بسط
$\frac{1}{(1-x)^k}$
برابر است با
$\binom{r+k-1}{r}$
پس ضریب
$x^{20}$
در عبارت داده شده برابر است با:
$$\binom{13}{11}$$
